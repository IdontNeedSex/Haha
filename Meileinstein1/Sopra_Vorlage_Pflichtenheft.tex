\documentclass{uulm-assignment}
\usepackage{import}
\usepackage{tabularx}
\usepackage{listings}
\usepackage{hyperref}
\usepackage{CJKutf8}
\usepackage{graphicx}

\setboolean{showsolutions}{false}

\ifthenelse{\boolean{showsolutions}}{
    \newcommand\mitloesung{1}%
}{
    \newcommand\mitloesung{0}%
}

\hypersetup{colorlinks=false,urlcolor=uulm-in}

\faculty{Institut für Softwaretechnik und Programmiersprachen\hspace{0.05cm}}
\course{Softwaregrundprojekt}
\semester{\hspace{0.05cm} WiSe 2022/23}
\supervisor{\hspace{7.9cm} Prof. Dr. Thomas Thüm, Sabrina Böhm, Valentin Kolb, Pascal Schiessle}

\assignmenttype{}
\assignmentno{}
\title{Pflichtenheft - Battle of the Centerländ}

\newcommand{\game}{Battle of the Centerländ}

\begin{document}

    \maketitle

    \section{Allgemeine Einleitung}

    \subsection{Einleitung}
    Hier kommen die Einleitung rein.
    
    \subsection{Motivation}
    Hier kommen die Motivation rein.
    \subsection{Vision}
    Hier kommen die Vision rein.
    \subsection{Projektkontext}
    Hier kommen die Projektkontext rein.
    \section{Allgemeine Fachwissen}

   Allgemeine Fachwissen für \emph{\game} (\href{https://de.wikipedia.org/wiki/Robo_Rally}{Link zu Wikipedia}).
   
   \begin{tabularx}{\textwidth}{|l|X |} \hline
        \textbf{Begriff} & \textbf{Charakter} \\
        \hline
        BESCHREIBUNG: &  \\
        \hline
        ASPEKT: &
        \\
        \hline
        BEMERKUNG: & \\
        \hline
        BEISPIEL: & +\\
        \hline
    \end{tabularx}
    
    \section{Spezifische Anforderungen}

    Dieser Abschnitt enthält alle spezifischen Anforderungen an das System. Er bietet eine detaillierte Beschreibung des Systems und seiner Funktionen.

    \subsection{Akteure und Funktionale Anforderungen}

    Dieser Abschnitt enthält alle Anforderungen, die die grundlegenden Aktionen des Softwaresystems spezifizieren.

    \begin{tabularx}{\textwidth}{|l|X |} \hline
        \textbf{ID} & \textbf{FA1} \\
        \hline
        TITEL: & Hauptmenü \\
        \hline
        BESCHREIBUNG: & Nach dem Anwendungsstart wird dem Spieler das Hauptmenü angezeigt. Der Spieler kann folgende Aktionen im Hauptmenü ausführen:
        \begin{itemize}
            \item Benutzer-Client oder KI-Client wählen
            \item Spiel starten und auf den Spielbildschirm wechseln
            \item Sprache wählen 
            \item Die Anwendung beenden
        \end{itemize}
        \\
        \hline
        BEGRÜNDUNG: & Nachdem der Benutzer die Anwendung gestartet hat, soll nicht direkt der Spielbildschirm erscheinen.
        Der Benutzer soll die Möglichkeit haben das Spiel zu starten und das Spielmodus wählen sobald es vom Benutzer gewünscht ist.\\
        \hline
        PRIORITÄT: & +++++ \\
        \hline
        ABHÄNGIGKEITEN: & FA2\\
         \hline
         AKTEUR: & Hauptmenü 
         \\
        \hline        
    \end{tabularx}

    \begin{tabularx}{\textwidth}{|l|X |} \hline
        \textbf{ID} & \textbf{FA2} \\
        \hline
        TITEL: & Spielbildschirm \\
        \hline
        BESCHREIBUNG: & Der Spieler bekommt auf dem Spielbildschirm den aktuellen Spielzustand angezeigt.
        \\
        \hline
        BEGRÜNDUNG: & Es muss einen Bildschirm geben, auf welchem der Spielzustand angezeigt wird, da das Spiel
        sonst nicht spielbar ist. \\
        \hline
        PRIORITÄT: & ++ \\
        \hline
        ABHÄNGIGKEITEN: & FA1, FA4, FA5,, FA15, FA16, FA18, FA19, FA21, FA22, FA23, FA25, FA28, FA30, FA34, FA37\\
        \hline 
        AKTEUR: & Visuell Darsteller
        \\
        \hline
    \end{tabularx}

    \begin{tabularx}{\textwidth}{|l|X |} \hline
        \textbf{ID} & \textbf{FA3} \\
        \hline
        TITEL: & Spielfeld erstellen \\
        \hline
        BESCHREIBUNG: & Das Spielfeld ist ein kartesisches Raster x * y (schachbrettartige Felder) und mit dem Editor kann man solche Spielbretter erstellen und in einer Board-Konfiguration speichern. 
        \\
        \hline
        BEGRÜNDUNG: & Es muss ein für zwei bis sechs Spielern sichtbares Spielfeld geben, damit sie spielen können.\\
        \hline
        ABHÄNGIGKEITEN: & FA4, FA5, FA6\\
        \hline
        AKTEUR: & Spielbrett Ersteller
        \\
        \hline
    \end{tabularx}

    \begin{tabularx}{\textwidth}{|l|X |} \hline
        \textbf{ID} & \textbf{FA4} \\
        \hline
        TITEL: & Spielfeld \\
        \hline
        BESCHREIBUNG: & Ein Spielfeld besteht aus verschiedenen Arten von Feldern, die von Charakteren betretbar sein können oder nicht. 
        \\
        \hline
        BEGRÜNDUNG: & Es sollte Eigenschaften für das Spielfeld geben. \\
        \hline
        PRIORITÄT: & ++\\
        \hline
        ABHÄNGIGKEITEN: & FA2, FA3, FA5, FA6, FA16, FA28, FA40\\
        \hline
        AKTEUR: & Spielbrett
        \\
        \hline
    \end{tabularx}

    \begin{tabularx}{\textwidth}{|l|X |} \hline
        \textbf{ID} & \textbf{FA5} \\
        \hline
        TITEL: &  Feld\\
        \hline
        BESCHREIBUNG: & Es sollte verschiedene Arten von Feldern geben, die verschiedene Eigenschaften haben. Ein Feld, auf dem ein anderer Charakter steht, ist für einen Charakter trotzdem betretbar, da Charaktere sich gegenseitig schieben können. 
        \\
        \hline
        BEGRÜNDUNG: & Der Charakter soll sich auf ein Feld befinden.\\
        \hline
        PRIORITÄT: & +\\
        \hline
        ABHÄNGIGKEITEN: & FA4, FA6, FA14, FA23, FA28, FA30, FA40\\
        \hline
        AKTEUR: & Komponenten des Spielbretts
        \\
        \hline
    \end{tabularx}
    
    \begin{tabularx}{\textwidth}{|l|X |} \hline
        \textbf{ID} & \textbf{FA6} \\
        \hline
        TITEL: &  Eigenschaften des Feldes \\
        \hline
        BESCHREIBUNG: & Die Eigenschaft des Feldes sind Start, Gras, Loch, Fluss, Lembas, Check point, Saurons Auge.
        \\
        \hline
        BEGRÜNDUNG: & Das Spielbrett soll diese Eigenschaft von Feldern enthalten. \\
        \hline
        PRIORITÄT: & +\\
        \hline
        ABHÄNGIGKEITEN: & FA7, FA8, FA9, FA10, FA11, FA12, FA13, FA28, FA40\\
        \hline
        AKTEUR: & Liste von Eigenschaften des Spielfelds \\
        \hline
    \end{tabularx}
    
    \begin{tabularx}{\textwidth}{|l|X |} \hline
        \textbf{ID} & \textbf{FA7} \\
        \hline
        TITEL: &  Feldart: Start\\
        \hline
        BESCHREIBUNG: & Auf dem Startfeld werden die Charaktere der Spieler:in beim Starten des
        Spiels und beim Wiederbeleben platziert. Es werden immer zwei bis sechs Startfelder auf dem Spielbrett platziert, je nachdem wie viel in der Board-Konfig angegeben
wurden, wobei je nach Spielendenanzahl nicht alle besetzt werden. Die Anzahl der
Startfelder gibt somit die maximale Spielendenanzahl an. Stirbt ein Charakter,
wird er wiederbelebt und startet von seinem Startfeld aus.
        \\
        \hline
        BEGRÜNDUNG: & Zu Beginn des Spiels sollte ein Feld für die Charakter geben. \\
        \hline
        PRIORITÄT: & +\\
        \hline
        ABHÄNGIGKEITEN: & FA4, FA5, FA6, FA16, FA38 \\
        \hline
        AKTEUR: &  Eigenschaft des Spielfelds\\
        \hline
    \end{tabularx}
    
    \begin{tabularx}{\textwidth}{|l|X |} \hline
        \textbf{ID} & \textbf{FA8} \\
        \hline
        TITEL: &  Feldart: Gras\\
        \hline
        BESCHREIBUNG: & Ein Grasfeld beschreibt ein betretbares Feld, das den Hauptuntergrund des Spielbretts ausmacht und keine Funktionalität beinhaltet.
        \\
        \hline
        BEGRÜNDUNG: & Ein Feldart ohne Einfluss auf das Spiel\\
        \hline
        PRIORITÄT: & +\\
        \hline
        ABHÄNGIGKEITEN: & FA4, FA5, FA6, FA40\\
        \hline
        AKTEUR: &  Eigenschaft des Spielfelds\\
        \hline
    \end{tabularx}
    
    \begin{tabularx}{\textwidth}{|l|X |} \hline
        \textbf{ID} & \textbf{FA9} \\
        \hline
        TITEL: &  Feldart: Loch\\
        \hline
        BESCHREIBUNG: & Auf dem Spielbrett sind eine gewisse Anzahl an Löchern, in die ein Charakter beim Fortbewegen auf dem Spielbrett fallen kann. Ein Charakter kann auch von
einem anderen Charakter in ein Loch geschoben werden und wird dann zurück auf das eigene Startfeld platziert.
        \\
        \hline
        BEGRÜNDUNG: & Ein Feldart, dass der Charakter auf diesem Feld sterben kann.\\
        \hline
        PRIORITÄT: & +\\
        \hline
        ABHÄNGIGKEITEN: & FA4, FA5, FA6, FA27, FA37,FA40\\
        \hline
        AKTEUR: & Eigenschaft des Spielfelds\\
        \hline
    \end{tabularx}
    
    \begin{tabularx}{\textwidth}{|l|X |} \hline
        \textbf{ID} & \textbf{FA10} \\
        \hline
        TITEL: &  Feldart: Fluss\\
        \hline
        BESCHREIBUNG: & Ein Fluss besteht aus mehreren aneinandergereihten Flussfeldern. Ein Fluss darf nicht diagonal verlaufen. Flussfelder besitzen eine Richtung. Endet der Zug eines
Charakters auf einem Flussfeld, wird er am Ende einer Zugrunde in Flussrichtung
zwei Felder weit fortbewegt
        \\
        \hline
        BEGRÜNDUNG: & Transportmittel für den Charakter\\
        \hline
        PRIORITÄT: & +\\
        \hline
        ABHÄNGIGKEITEN: & FA4, FA5, FA6, FA29, FA30, FA40\\
        \hline
        AKTEUR: & Eigenschaft des Spielfelds \\
        \hline
    \end{tabularx}
    
    \begin{tabularx}{\textwidth}{|l|X |} \hline
        \textbf{ID} & \textbf{FA11} \\
        \hline
        TITEL: &  Feldart: Lembas\\
        \hline
        BESCHREIBUNG: & Auf dem Spielbrett sind Lembasfelder zu finden, die auch betretbare Felder darstellen. Wenn ein Charakter ein Lembasfeld betritt, sammelt er ein Stück
Lembasbrot automatisch ein. Ein Lembasfeld kann noch Lembasbrot enthalten
oder schon leer sein.
        \\
        \hline
        BEGRÜNDUNG: & Spezielle Feldart für das Spiel \\
        \hline
        PRIORITÄT: & +\\
        \hline
        ABHÄNGIGKEITEN: & FA4, FA5, FA6, FA22, FA23, FA25, FA40\\
        \hline
        AKTEUR: & Eigenschaft des Spielfelds\\
        \hline
    \end{tabularx}
    
    \begin{tabularx}{\textwidth}{|l|X |} \hline
        \textbf{ID} & \textbf{FA12} \\
        \hline
        TITEL: &  Feldart: Checkpoint\\
        \hline
        BESCHREIBUNG: & In jeder Spielpartie muss es mindestens zwei Checkpointfelder geben, die von den Charakteren erreicht werden müssen, um das Spiel zu gewinnen. Diese
Checkpoints sind sichtbar durchnummeriert und stellen auch betretbare Felder dar.
Ein Checkpoint gilt als erreicht, wenn ein Charakter am Ende einer Zugreihenfolge
auf dem Checkpoint steht.
        \\
        \hline
        BEGRÜNDUNG: & Ziel des Spiels, dass der Charakter die Reihenfolge von gegebenen Checkpoints erreicht.
        \\
        \hline
        PRIORITÄT: & +\\
        \hline
        ABHÄNGIGKEITEN: & FA4, FA5, FA6, FA39, FA40\\
        \hline
        AKTEUR: & Eigenschaft des Spielfelds \\
        \hline
    \end{tabularx}
    
    \begin{tabularx}{\textwidth}{|l|X |} \hline
        \textbf{ID} & \textbf{FA13} \\
        \hline
        TITEL: & Feldart: Saurons Auge \\
        \hline
        BESCHREIBUNG: &  Das Augefeld gibt es genau einmal pro Spielbrett. Es stellt ein besetztes Feld dar und kann nicht von Charakteren betreten werden. Durch dieses Feld kann
wie bei einer Wand nicht "durchgeschossen" werden, bzw. kein Charakter kann
hindurch einen anderen Charakter verwunden. Dieses Feld ist für die Berechnung
der Spielreihenfolge pro Zugrunde wichtig.
        \\
        \hline
        BEGRÜNDUNG: & Spezielle Feldart für das Spiel\\
        \hline
        PRIORITÄT: & +\\
        \hline
        ABHÄNGIGKEITEN: & FA4, FA5, FA6, FA35, FA40\\
        \hline
        AKTEUR: & Eigenschaft des Spielfelds
        \\
        \hline
    \end{tabularx}
    
    \begin{tabularx}{\textwidth}{|l|X |} \hline
        \textbf{ID} & \textbf{FA14} \\
        \hline
        TITEL: &  Weg von Feld A zu Feld B\\
        \hline
        BESCHREIBUNG: & Die Entfernung zwischen zwei Feldern A und B ist definiert als die minimale Anzahl von aufeinanderfolgenden Schritten auf alle 4 Nachbarfelder, um von A nach B zu gelangen. Die Metrik ist bekannt als die Manhatten Distanz oder auch Taxi-/Cityblock Metrik. Wände auf dem Spielbrett müssen natürlich mit berücksichtigt werden, durch die nicht hindurch gelaufen werden kann.
        \\
        \hline
        BEGRÜNDUNG: & Der Weg von Feld A zu Feld B muss definiert werden.\\
        \hline
        PRIORITÄT: & +\\
        \hline
        ABHÄNGIGKEITEN: & FA5, FA16 \\
        \hline
        AKTEUR: & Weg\\
        \hline
    \end{tabularx}
    
    \begin{tabularx}{\textwidth}{|l|X |} \hline
        \textbf{ID} & \textbf{FA15} \\
        \hline
        TITEL: &  Charakterwahl\\
        \hline
        BESCHREIBUNG: & Jeder Spieler wählt zu Beginn des Spiels einen Charakter. Der gewählte Charakter repräsentiert die spielende Person visuell im Spiel. Da die Charaktere keine spezifischen Vor- oder Nachteile mit sich bringen, hat die Wahl keinen Einfluss auf den Spielverlauf. Zu Beginn bei der Charakterwahl erhält jeder/jede Spieler:in zwei zufällige Charaktere zur Auswahl, aus denen einer gewählt wird. Ein Charakter darf nicht mehreren Spieler:innen gleichzeitig zur Auswahl angeboten werden, das heißt, dass zwei Spielende pro Partie nie den gleichen Charakter spielen dürfen.
        \\
        \hline
        BEGRÜNDUNG: &  Der Spieler darf den Charakter wählen, den er mag.\\
        \hline
        PRIORITÄT: & +\\
        \hline
        ABHÄNGIGKEITEN: & FA1, FA16\\
        \hline
        AKTEUR: & \\
        \hline
    \end{tabularx}
    
    \begin{tabularx}{\textwidth}{|l|X |} \hline
        \textbf{ID} & \textbf{FA16} \\
        \hline
        TITEL: &  Charakter\\
        \hline
        BESCHREIBUNG: & Die Charakter haben keine spezifischen Vor- und Nachteilen während des Spiels, somit hat die Charakter keinen Einfluss auf den Spielverlauf. Die folgende Charakter sind: Frodo(Ring), Sam(Bratpfanne), Legolas(Bogen), Gimli(Axt), Gandalf(Zauberstab), Aragon(Schwert), Gollum(Fisch), Galadriel(Dolch), Boromir(Horn), Baumbart(Baumstamm), Merry(Rakete), Pippin(Pfeife), Arwen(Kette).
\newline  
Die Charaktere sollen auf dem Spielbrett klar unterscheidbar dargestellt werden. Die dem Charakter zugehörige Waffe soll sichtbar am Charakter angebracht sein. Während des Spiels ist es möglich andere Charaktere zu verwunden oder zu töten, was pro Charakter unterschiedlich ausgeführt wird, z.B. Gollum wirft einen Fisch auf den Charakter und Frodo benutzt die magische Wirkung des Rings um den Chrakter zu verwunden.
        \\
        \hline
        BEGRÜNDUNG: & Es soll verschiedene Charaktere geben, die für den Spieler zum Wählen verfügbar sind. \\
        \hline
        PRIORITÄT: & +++\\
        \hline
        ABHÄNGIGKEITEN: & FA14, FA15, FA18, FA21, FA23, FA25 , FA26, FA30, FA35, FA37, FA38, FA39, FA41\\
        \hline
        AKTEUR: & Charakter\\
        \hline
    \end{tabularx}
    
    \begin{tabularx}{\textwidth}{|l|X |} \hline
        \textbf{ID} & \textbf{FA17} \\
        \hline
        TITEL: &  Kartenset\\
        \hline
        BESCHREIBUNG: & Jedem/Jeder Spieler:in bekommen ein gleiche Kartenset aus 20 Karten.
        \\
        \hline
        BEGRÜNDUNG: & Um sich auf dem Spielbrett fortzubewegen, erhält jeder Spielende Karten mit Anweisungen, die er oder sie verdeckt legen und Zug um Zug spielen muss.\\
        \hline
        PRIORITÄT: & +\\
        \hline
        ABHÄNGIGKEITEN: & FA16, FA18\\
        \hline
        AKTEUR: & Liste von Karten\\
        \hline
    \end{tabularx}
    
    \begin{tabularx}{\textwidth}{|l|X |} \hline
        \textbf{ID} & \textbf{FA18} \\
        \hline
        TITEL: &  Karten\\
        \hline
        BESCHREIBUNG: &  Es gibt verschiedene Karten mit verschiedene Anzahl und Anweisung. Mit der Anweisung von Karten kann der/die Spieler:in auf dem Spielbrett bewegen. Der Charakter kann x Felder geradeaus(Move x) in Blickrichtung des Charakters bewegen. Er kann rückwärts bewegen, 180 Grad oder 90 Grad im/gegen Uhrzeigersinn umdrehen, führt die vorherige Anweisung erneut aus, fügt dem Spieler ein Lembas zu seinem Vorrat hinzu.
        \\
        \hline
        BEGRÜNDUNG: & Der Charakter kann nur bewegen mit der Anweisung von Karten.\\
        \hline
        PRIORITÄT: & +\\
        \hline
        ABHÄNGIGKEITEN: & FA14, FA16, FA17, FA19, FA24, FA31, FA33, FA34, FA36\\
        \hline
        AKTEUR: & Bewegungsveränderung\\
        \hline
    \end{tabularx}
    
    \begin{tabularx}{\textwidth}{|l|X |} \hline
        \textbf{ID} & \textbf{FA19} \\
        \hline
        TITEL: &  Nachziehstapel\\
        \hline
        BESCHREIBUNG: & Die Kartenset liegen am Anfang auf dem Nachziehstapel, von welchen die Handkarten gezogen werden. Noch nicht gezogene Karten liegen auf dem Nachziehstapel, während schon gespielte Karten auf den Ablagestapel wandern.
        \\
        \hline
        BEGRÜNDUNG: & Nachziehstapel dient als Schnittstelle zwischen Kartenset und Handkarten, dass der/die Spieler:in am Anfang nicht alle Karten haben, sondern ziehen muss.\\
        \hline
        PRIORITÄT: & +\\
        \hline
        ABHÄNGIGKEITEN: & FA18, FA34, FA38\\
        \hline
        AKTEUR: & Nachziehstapel\\
        \hline
    \end{tabularx}
    
    \begin{tabularx}{\textwidth}{|l|X |} \hline
        \textbf{ID} & \textbf{FA20} \\
        \hline
        TITEL: &  Checkpoints erstellen\\
        \hline
        BESCHREIBUNG: & 
         Die Anzahl der Checkpoints ergibt sich durch die festgelegten Board-Konfiguration und es dürfen beliebig viele Checkpoints definiert werden. Die Anzahl an schon erreichten Checkpoints soll jeweils pro Spieler grafisch dargestellt werden. 
        \\
        \hline
        BEGRÜNDUNG: & Die Anzahl soll festgelegt und definiert werden.\\
        \hline
        PRIORITÄT: & +\\
        \hline
        ABHÄNGIGKEITEN: & FA40\\
        \hline
        AKTEUR: & Spielbrett Ersteller\\
        \hline
    \end{tabularx}
    
    \begin{tabularx}{\textwidth}{|l|X |} \hline
        \textbf{ID} & \textbf{FA21} \\
        \hline
        TITEL: & Anzahl an Leben \\
        \hline
        BESCHREIBUNG: & Jeder Charakter hat zu Beginn einer Spielpartie drei Leben. Wenn ein Charakter von einem anderen
verwundet wird, verliert er ein Leben. Das Verlieren eines Lebens bedeutet, dass der Charakter in der Planungsphase nun pro verlorenem Leben eine Handkarte weniger zur Verfügung hat.
\newline Wenn ein Charakter alle drei Leben verliert, stirbt er und setzt für die restliche Runde aus. Wenn er wiederbelebt wird, erhält er seine drei Leben zurück und darf in der Planungsphase wieder alle neun Handkarten ziehen.
\newline Die aktuelle Anzahl an Leben pro Spielenden soll graphisch dargestellt werden und für alle anderen Spielenden ebenso sichtbar sein .
        \\
        \hline
        BEGRÜNDUNG: & Der Charakter hat ein Anzahl an Leben, damit er auch sterben kann.\\
        \hline
        PRIORITÄT: & +\\
        \hline
        ABHÄNGIGKEITEN: &  FA26, FA27, FA37\\
        \hline
        AKTEUR: & Leben des Charakters\\
        \hline
    \end{tabularx}
    
    \begin{tabularx}{\textwidth}{|l|X |} \hline
        \textbf{ID} & \textbf{FA22} \\
        \hline
        TITEL: &  Lembas Highlight\\
        \hline
        BESCHREIBUNG: & Board-Konfiguration erstellt Lembasfelder auf dem Spielbrett. Die Lembasfelder werden sichtbar markiert, wenn das Lembasbrot noch auf dem Feld liegt oder schon eingesammelt wurde.  Wenn ein Charakter ein Lembasfeld betritt, sammelt er das darauf liegende Lembas sofort ein und fügt sie seinem Bestand hinzu.
        \\
        \hline
        BEGRÜNDUNG: & Es soll sichtbar für den Spieler machen, dass er auf ein Lembasfeld betritt.\\
        \hline
        PRIORITÄT: & +\\
        \hline
        ABHÄNGIGKEITEN: & FA11, FA23 \\
        \hline
        AKTEUR: & \\
        \hline
    \end{tabularx}
    
    \begin{tabularx}{\textwidth}{|l|X |} \hline
        \textbf{ID} & \textbf{FA23} \\
        \hline
        TITEL: &  Lembsbrot\\
        \hline
        BESCHREIBUNG: & Es gibt eine bestimmte Anzahl an Lembasbrot, das jeder Spieler zu Beginn im Bestand hat. Diese Anzahl wird in der Partie-Konfig eingestellt. Die aktuelle Anzahl an Lembas pro Spielenden soll graphisch dargestellt werden und für alle anderen Spielenden ebenso sichtbar sein.
        \\
        \hline
        BEGRÜNDUNG: & Ein Gegendstand, dass der Charakter auf Lembas einsammeln kann.\\
        \hline
        PRIORITÄT: & +\\
        \hline
        ABHÄNGIGKEITEN: & FA11, FA23, FA25\\
        \hline
        AKTEUR: & Points für das Angriff\\
        \hline
    \end{tabularx}
    
    \begin{tabularx}{\textwidth}{|l|X |} \hline
        \textbf{ID} & \textbf{FA24} \\
        \hline
        TITEL: &  Schieben\\
        \hline
        BESCHREIBUNG: & Ein Charakter kann einen anderen Charakter über das Spielbrett schieben, wenn die Charaktere in der Bewegungsphase nacheinander ihre Karten spielen. Wenn eine Wand im Weg ist, wird nicht weitergeschoben. Ein Charakter kann, der z.B. auf einen angestrebten Checkpoint steht, von einem anderen Charakter der nach ihm an der Reihe ist vom Checkpoint wieder runtergeschoben werden. \newline Außerdem kann ein Charakter einen anderen in ein Loch oder vom Spielbrett schieben. Wenn dies passiert ist, stirbt der Charakter und alle seine verbleibenden Züge
werden ausgesetzt, bis er in der neuen Runde auf seinem Startfeld wiederbelebt wird. Das Schieben
ist natürlich vorwärts und rückwärts gleichbedeutend und ändert von keinen beteiligten Charakteren
die Blickrichtung. Beim Laufen und auch Schieben, also in der Bewegungsphase, werden Flüsse ignoriert. Dies bedeutet,
dass ein Charakter seine Blickrichtung nicht ändert, falls er auf eine Flusskurve läuft.
        \\
        \hline
        BEGRÜNDUNG: & Der Charakter kann sich gegenseitig schieben, damit das Spiel auch interessanter wird. \\
        \hline
        PRIORITÄT: & +\\
        \hline
        ABHÄNGIGKEITEN: & FA16\\
        \hline
        AKTEUR: & Fähigkeit des Charakters\\
        \hline
    \end{tabularx}
    
    \begin{tabularx}{\textwidth}{|l|X |} \hline
        \textbf{ID} & \textbf{FA25} \\
        \hline
        TITEL: &  Waffe\\
        \hline
        BESCHREIBUNG: & Wenn der Charakter genug Lembas aufgesammelt hat, kann ein Charakter mit seiner Waffe im
Weg stehende Charaktere verwunden oder beschießen. Im Weg stehen bedeutet hier die gerade
Blickrichtung des Charakters über das Spielbrett hinweg. Durch Wände und Saurons Auge kann
nicht "durchgeschossen" werden. Über Flüsse und Löcher kann hinweg geschossen werden.\newline
Es wird immer am Ende einer Zugrunde, in der Aktionsphase, geschossen (also wenn alle Spieler:innen
für den aktuellen Zug gelaufen sind) wenn ein Charakter getroffen werden könnte und ein spielende
Person genug Lembas hat. Man kann also nicht entscheiden, ob man schießt oder nicht. Wie viel
Lembas es kostet einmal zu schießen wird in der Partie-Konfig festgelegt.
        \\
        \hline
        BEGRÜNDUNG: & Mit der Waffe kann der Charakter Schaden an anderen Charakteren verursachen. \\
        \hline
        PRIORITÄT: & +\\
        \hline
        ABHÄNGIGKEITEN: & FA16, FA23, FA26\\
        \hline
        AKTEUR: & Fähigkeit des Charakters\\
        \hline
    \end{tabularx}
    
    \begin{tabularx}{\textwidth}{|l|X |} \hline
        \textbf{ID} & \textbf{FA26} \\
        \hline
        TITEL: & Schaden \\
        \hline
        BESCHREIBUNG: & Wenn ein Charakter einen anderen Charakter verwundet, verliert der/die Getroffene ein Leben und ihm/ihr
wird eine Handkarte gesperrt. Dies bedeutet, dass dem/der Getroffenen in den folgenden Runden
eine Karte weniger ausgeteilt wird. So repräsentiert die Anzahl der Handkarten, die man weniger
bekommt, den Schaden, den man durch das verwundet werden bekommt.
        \\
        \hline
        BEGRÜNDUNG: & Schaden von dem Angriff, wenn der Charakter genug Lembasbrot hat. \\
        \hline
        PRIORITÄT: & +\\
        \hline
        ABHÄNGIGKEITEN: & FA21, FA25, FA28, FA37 \\
        \hline
        AKTEUR: & Schaden verringert das Leben des Charakters\\
        \hline
    \end{tabularx}
    
    \begin{tabularx}{\textwidth}{|l|X |} \hline
        \textbf{ID} & \textbf{FA27} \\
        \hline
        TITEL: &  Eigenschaften des Lochs\\
        \hline
        BESCHREIBUNG: & Der Charakter kann ins Loch reinfallen. Nach dem Reinfallen stirbt der Charakter und setzt für die aktuelle
Runde aus. Er wird zum Start der neuen Runde wiederbelebt. Löcher müssen so dargestellt werden,
dass ein Unterschied zu jeder anderen Feldart zu sehen ist.
        \\
        \hline
        BEGRÜNDUNG: & Das Loch dient als Feld, womit der Charakter anderen Charakter darein schieben und töten kann. \\
        \hline
        PRIORITÄT: & +\\
        \hline
        ABHÄNGIGKEITEN: & FA9, FA37 \\
        \hline
        AKTEUR: & \\
        \hline
    \end{tabularx}
    
    \begin{tabularx}{\textwidth}{|l|X |} \hline
        \textbf{ID} & \textbf{FA28} \\
        \hline
        TITEL: &  Wände\\
        \hline
        BESCHREIBUNG: & Auf dem Spielbrett gibt es Wände, die immer zwischen zwei Feldern stehen, aber nicht zwischen zwei Flussfeldern. Wände versperren den Charakteren den Weg und sie können nicht überlaufen werden. Wenn ein
Charakter auf einem Feld steht und in Richtung der Wand schaut und beispielsweise noch einen
Schritt geradeaus machen müsste passiert nichts und der Charakter bleibt weiterhin auf diesem Feld
stehen. Außerdem kann ein Charakter einen anderen nicht durch eine Wand schieben
        \\
        \hline
        BEGRÜNDUNG: & Wände dient als das Feld, das der Angriff oder die Bewegung des Charakters blockiert. \\
        \hline
        PRIORITÄT: & +\\
        \hline
        ABHÄNGIGKEITEN: & FA5, FA40\\
        \hline
        AKTEUR: & Eigenschaft des Spielfelds  \\
        \hline
    \end{tabularx}
    
    \begin{tabularx}{\textwidth}{|l|X |} \hline
        \textbf{ID} & \textbf{FA29} \\
        \hline
        TITEL: & Flüsse erstellen\\
        \hline
        BESCHREIBUNG: &  Ein Fluss darf keine Abzweigungen oder Flusskreuzungen haben. Ebenfalls hat der Fluss eine konsistente
Fließrichtung mit Start- und Endpunkt, die durch die aneinandergereihten einzelnen Felder dargestellt wird. Die Fließrichtung wird aus der Board-Konfig herausgelesen. 
        \\
        \hline
        BEGRÜNDUNG: & Die Eigenschaft des Flusses soll erstellt werden.\\
        \hline
        PRIORITÄT: & +\\
        \hline
        ABHÄNGIGKEITEN: & FA5, FA10, FA30, FA40\\
        \hline
        AKTEUR: & Board Ersteller\\
        \hline
    \end{tabularx}
    
    \begin{tabularx}{\textwidth}{|l|X |} \hline
        \textbf{ID} & \textbf{FA30} \\
        \hline
        TITEL: & Bewegung auf Fluss\\
        \hline
        BESCHREIBUNG: & In der Aktionsphase, wenn die Spielbrettaktionen abgehandelt werden
und der Fluss fließt, bewegt sich der Charakter auf Flussfeldern mit den Flussfeldeigenschaften und die Blickrichtung ändert sich in die Flussrichtung. Wenn ein Charakter noch ein oder zwei Flussfelder transportiert werden sollte, der Fluss aber
endet, wird er immer auf das Feld nach dem letzten Flussfeld geschwemmt.
        \\
        \hline
        BEGRÜNDUNG: & Die Blickrichtung sollte ändern, damit der Charakter strategisch etwas planen kann.\\
        \hline
        PRIORITÄT: & +\\
        \hline
        ABHÄNGIGKEITEN: & FA16, FA29\\
        \hline
        AKTEUR: & Bewegungsveränderung \\
        \hline
    \end{tabularx}
    
    \begin{tabularx}{\textwidth}{|l|X |} \hline
        \textbf{ID} & \textbf{FA31} \\
        \hline
        TITEL: & Karten ziehen\\
        \hline
        BESCHREIBUNG: & Der Spieler zieht eine zufällige Karte aus der Nachziehstapel.
        \\
        \hline
        BEGRÜNDUNG: & Der Spieler hat neun Karten zu Beginn und muss weitere Karten ziehen.\\
        \hline
        PRIORITÄT: & +\\
        \hline
        ABHÄNGIGKEITEN: & FA14, FA16, FA18, FA19\\
        \hline
        AKTEUR: & Transportieren Karten von Nachziehstapel zu Handkarten \\
        \hline
    \end{tabularx}
    
    \begin{tabularx}{\textwidth}{|l|X |} \hline
        \textbf{ID} & \textbf{FA32} \\
        \hline
        TITEL: & Phase wechseln\\
        \hline
        BESCHREIBUNG: & Das Spiel läuft in Runden ab, die jeweils aus zwei Phasen. Nach der ersten Phase soll es zu den zweiten Phase gewechselt werden.
        \\
        \hline
        BEGRÜNDUNG: & Ohne Phase wechseln kann das Spiel nicht fortsetzen.\\
        \hline
        PRIORITÄT: & +\\
        \hline
        ABHÄNGIGKEITEN: & FA43\\
        \hline
        AKTEUR: & \\
        \hline
    \end{tabularx}
    
    \begin{tabularx}{\textwidth}{|l|X |} \hline
        \textbf{ID} & \textbf{FA33} \\
        \hline
        TITEL: & Karten verdecken\\
        \hline
        BESCHREIBUNG: & Die Handkarten sollen vor den anderen Spielern verdeckt werden 
        \\
        \hline
        BEGRÜNDUNG: & Der Spieler darf die Handkarten von anderen Spielern nicht sehen.\\
        \hline
        PRIORITÄT: & +\\
        \hline
        ABHÄNGIGKEITEN: & FA18 \\
        \hline
        AKTEUR: & \\
        \hline
    \end{tabularx}
    

    
    \begin{tabularx}{\textwidth}{|l|X |} \hline
        \textbf{ID} & \textbf{FA34} \\
        \hline
        TITEL: & Ablagestapel\\
        \hline
        BESCHREIBUNG: & Die Karten, die nicht eingeloggt wurden, wandern auf den Ablagestapel.  Falls der Nachziehstapel nicht genügend Karten enthält, werden dem
Spieler zunächst alle übrigen Karten des Nachziehstapels ausgeteilt. Anschließend werden alle Karten
des Ablagestapels gemischt und zum neuen Nachziehstapel, von dem nun die restlichen Handkarten
gezogen werden, bis der Spieler auf seine Anzahl an Handkarten kommt
        \\
        \hline
        BEGRÜNDUNG: & Fall der Spieler nicht mehr Karten auf dem Nachziehstapel hat, kann er nach dem Mischen vom Nachziehstapel weiter spielen.\\
        \hline
        PRIORITÄT: & +\\
        \hline
        ABHÄNGIGKEITEN: & FA18, FA38\\
        \hline
        AKTEUR: & Ablagestapel\\
        \hline
    \end{tabularx}
    
    \begin{tabularx}{\textwidth}{|l|X |} \hline
        \textbf{ID} & \textbf{FA35} \\
        \hline
        TITEL: & Spielreihenfolge\\
        \hline
        BESCHREIBUNG: & Der Server bestimmt die Spielreihenfolge der ersten Zugrunde und die Bewegungsphase startet. Zu Beginn jeder Zugrunde wird die Spielerreihenfolge neu bestimmt. Es beginnt immer der Charakter,
der am nächsten an Saurons Auge ist.
        \\
        \hline
        BEGRÜNDUNG: & Der Server macht es deutlich, welcher Spieler als Erstes anfängt.\\
        \hline
        PRIORITÄT: & +\\
        \hline
        ABHÄNGIGKEITEN: & FA13, FA16\\
        \hline
        AKTEUR: & Spielreihenfolge\\
        \hline
    \end{tabularx}
    
    \begin{tabularx}{\textwidth}{|l|X |} \hline
        \textbf{ID} & \textbf{FA36} \\
        \hline
        TITEL: & Karten aufdecken\\
        \hline
        BESCHREIBUNG: & Nachdem die Reihenfolge der Spieler bestimmt wurde, werden die Karten der ersten Zugrunde Spieler
nach Spieler aufgedeckt. Nachdem die erste Karte des ersten Spielers aufgedeckt wurde, wird die
Aktion der Karte ausgeführt. 
        \\
        \hline
        BEGRÜNDUNG: & Die Anweisung der Karten wird nur nach der Aufdeckung der Karten aufgeführt.\\
        \hline
        PRIORITÄT: & +\\
        \hline
        ABHÄNGIGKEITEN: & FA18\\
        \hline
        AKTEUR: & \\
        \hline
    \end{tabularx}
    
    \begin{tabularx}{\textwidth}{|l|X |} \hline
        \textbf{ID} & \textbf{FA37} \\
        \hline
        TITEL: & Sterben\\
        \hline
        BESCHREIBUNG: &
Sterben kann ein Charakter, wenn er entweder vom Spielbrett läuft oder geschoben wird, in ein Loch
fällt oder geschoben wird, oder zum dritten Mal von einem anderen Charakter verwundet wird und
somit sein drittes Leben verliert. \newline
Wenn ein Charakter stirbt, verschwindet seine Charakterfigur direkt vom Spielbrett und er muss
für die aktuell zu spielende Runde in allen kommenden Zugrunden aussetzen. Alle seine übrigen
eingeloggten Züge werden verworfen und verdeckt auf seinen Ablagestapel gelegt.
        \\
        \hline
        BEGRÜNDUNG: & Nach dem Sterben kann der Spieler nicht spielen und er muss auf das Wiederbeleben warten.\\
        \hline
        PRIORITÄT: & +\\
        \hline
        ABHÄNGIGKEITEN: & FA16\\
        \hline
        AKTEUR: & Charakter Delete \\
        \hline
    \end{tabularx}
    
    \begin{tabularx}{\textwidth}{|l|X |} \hline
        \textbf{ID} & \textbf{FA38} \\
        \hline
        TITEL: & Wiederbeleben\\
        \hline
        BESCHREIBUNG: & Wenn neue Runde beginnt, wird der Charakter auf seinem Startfeld wiederbelebt. Das bedeutet, der Charakter wird wieder auf seine Anfangseinstellungen gesetzt, er hat wieder alle drei Leben, den eingestellten Startvorrat an Lembas, alle neun Handkarten und
seine Blickrichtung ist wie zu Beginn der Partie. Zusätzlich wandern alle 20 Karten des Kartensets
neu gemischt auf den Nachziehstapel. Die erreichten Checkpoints werden ihm aber nicht wieder
aberkannt, was bedeutet, dass sobald ein Checkpoint als erreicht gilt, das unveränderbar ist.
\newline Wenn auf dem Startfeld des Charakters, der am Anfang der nächsten Runde wiederbelebt werden
soll, gerade ein anderer Charakter steht, wird der wiederbelebte Charakter zufällig auf einem anderen
Startfeld wiederbelebt.
        \\
        \hline
        BEGRÜNDUNG: & Der Spieler kann nach dem Sterben wieder spielen. \\
        \hline
        PRIORITÄT: & +\\
        \hline
        ABHÄNGIGKEITEN: & FA16, FA18, FA19, FA34\\
        \hline
        AKTEUR: & Charakter Reset\\
        \hline
    \end{tabularx}
    
    \begin{tabularx}{\textwidth}{|l|X |} \hline
        \textbf{ID} & \textbf{FA39} \\
        \hline
        TITEL: & Bestimmung der siegenden Person \\
        \hline
        BESCHREIBUNG: & Der Spielende der zuerst alle Checkpoints abgelaufen hat, d.h. am Ende einer Zugrunde jeweils auf
allen Checkpoints in richtiger Reihenfolge war, hat gewonnen.
        \\
        \hline
        BEGRÜNDUNG: & Die Partie muss irgendwann zur Ende kommen und der Sieger der Partie wurde bestimmt.\\
        \hline
        PRIORITÄT: & +++\\
        \hline
        ABHÄNGIGKEITEN: & FA12, FA16 \\
        \hline
        AKTEUR: & Ende\\
        \hline
    \end{tabularx}
    
    
    \begin{tabularx}{\textwidth}{|l|X |} \hline
        \textbf{ID} & \textbf{FA40} \\
        \hline
        TITEL: & Board-Konfiguration \\
        \hline
        BESCHREIBUNG: & Board-Konfiguration enthält die Daten für das Spielbrett
        \\
        \hline
        BEGRÜNDUNG: & Die Daten von das Spielbrett soll ins Board-Konfiguration gespeichert werden.\\
        \hline
        PRIORITÄT: & +\\
        \hline
        ABHÄNGIGKEITEN: & FA4, FA5, FA6, FA20, FA22, FA29 \\
        \hline
        AKTEUR: & \\
        \hline
    \end{tabularx}
    
    \begin{tabularx}{\textwidth}{|l|X |} \hline
        \textbf{ID} & \textbf{FA41} \\
        \hline
        TITEL: &  Partie-Konfiguration\\
        \hline
        BESCHREIBUNG: & Fast alle Werte und Parameter, die im Spiel wichtig sind, sollen in der Partie-Konfiguration stehen.
        \\
        \hline
        BEGRÜNDUNG: & Das hat den Vorteil, dass man nachträglich noch viel Freiraum hat, um zu ändern, wie sich das Spiel
dann letztendlich anfühlt. All diese Parameter beeinflussen das Balancing des Spiels und es ist kaum
möglich vorab Werte zu wählen, für die das Spiel am besten “funktioniert”. Dadurch, dass alle Werte
aus einer Konfigurationsdatei eingelesen werden, kann man sie leicht ändern, während der Code
für die Spiellogik dadurch parametrisiert wird und stabil bleibt.  \\
        \hline
        PRIORITÄT: & +\\
        \hline
        ABHÄNGIGKEITEN: & FA16, FA17, FA18, FA21, FA23, FA26\\
        \hline
        AKTEUR: & \\
        \hline
    \end{tabularx}
    
    \begin{tabularx}{\textwidth}{|l|X |} \hline
        \textbf{ID} & \textbf{FA42} \\
        \hline
        TITEL: & Handkartenstapel\\
        \hline
        BESCHREIBUNG: & Nachdem die Karten vom Nachziehstapel gezogen wurden, kommen sie auf den Handstapel und der Spieler kann seine Karten ausspielen.
        \\
        \hline
        BEGRÜNDUNG: & Da der Spieler am Anfang nicht alle seine Karten erhält, soll seine verfügbare Karten auf den Handstapel liegen. \\
        \hline
        PRIORITÄT: & +\\
        \hline
        ABHÄNGIGKEITEN: & FA18, FA31\\
        \hline
        AKTEUR: & Phase\\
        \hline
    \end{tabularx}
    
    \begin{tabularx}{\textwidth}{|l|X |} \hline
        \textbf{ID} & \textbf{FA43} \\
        \hline
        TITEL: & Phase\\
        \hline
        BESCHREIBUNG: & Das Spiel besteht aus zwei Phasen. Die erste Phase
ist die Planungsphase, in der jeder Spielende aus den Handkarten fünf Züge einloggt. Die zweite
Phase ist die Rundenphase, die aus fünf Zugrunden besteht. Eine Zugrunde besteht wiederum aus
zwei Phasen, die Bewegungsphase und die Aktionsphase.
        \\
        \hline
        BEGRÜNDUNG: & Die Phase von dem Spiel soll eindeutig eingeteilt werden, wann der Spieler seine Karten spielen kann und wann der Charakter bewegt. \\
        \hline
        PRIORITÄT: & +\\
        \hline
        ABHÄNGIGKEITEN: & FA16, FA31, FA32 \\
        \hline
        AKTEUR: & Phase\\
        \hline
    \end{tabularx}
    

    \subsection{Nicht-funktionale Anforderungen}

    Dieser Abschnitt spezifiziert die nicht-funktionalen Anforderungen an das Softwaresystem.
    
    \begin{tabularx}{\textwidth}{|l|X |} \hline
        \textbf{ID} & \textbf{NFA1} \\
        \hline
        TITEL: & Kommunikation \\
        \hline
        BESCHREIBUNG: & Clients kommunizieren nur mit dem Server. Also findet keine Kommunikation zwischen den Clients selbst statt. \\
        \hline
        BEGRÜNDUNG: & Somit ist gewähleistet, dass der Server immer bescheid weiß. \\
        \hline
    \end{tabularx}
    
     \begin{tabularx}{\textwidth}{|l|X |} \hline
        \textbf{ID} & \textbf{NFA2} \\
        \hline
        TITEL: &  Server\\
        \hline
        BESCHREIBUNG: & Instanz welche eine Partie verwaltet und abwickelt.  Er ist verantwortlich die Spieleinstellungen zu Beginn des Spiels zu laden und die Spielregeln sowie das Spielbred zu verwalten.\\
        \hline
        BEGRÜNDUNG: &  Dadurch ist es möglich, das Spiel als im Mehrspielermodus zu gestalten.\\
        \hline
    \end{tabularx}

    \begin{tabularx}{\textwidth}{|l|X |} \hline
        \textbf{ID} & \textbf{NFA3} \\
        \hline
        TITEL: &  PartieKonfiguration\\
        \hline
        BESCHREIBUNG: &  Datei welche die konkreten Einstellungen einer Partie enthält\\
        \hline
        BEGRÜNDUNG: &  Bessere Übersichtlichkeit.\\
        \hline
    \end{tabularx}

    \begin{tabularx}{\textwidth}{|l|X |} \hline
        \textbf{ID} & \textbf{NFA4} \\
        \hline
        TITEL: &  Serverstart\\
        \hline
        BESCHREIBUNG: &  Ein Server muss nicht-interaktiv über die Kommandozeile in Form eines Docker-
Containers gestartet werden können \\
        \hline
        BEGRÜNDUNG: &  Anwender sollen nichts mt dem Serverstart zu tun haben und so soll ein schnellere Serververfügbarkeit gewährleistet werden.\\
        \hline
    \end{tabularx}

    \begin{tabularx}{\textwidth}{|l|X |} \hline
        \textbf{ID} & \textbf{NFA5} \\
        \hline
        TITEL: & Anzahl Clients \\
        \hline
        BESCHREIBUNG: &  Der Server ermöglicht es bis zu sechs Clients sich als MItspieler anzumelden.\\
        \hline
        BEGRÜNDUNG: &  So wird die Spielranzahl aus der Board-Konfig eingehalten.\\
        \hline
    \end{tabularx}

    \begin{tabularx}{\textwidth}{|l|X |} \hline
        \textbf{ID} & \textbf{NFA6} \\
        \hline
        TITEL: &  möglicher Spielstart\\
        \hline
        BESCHREIBUNG: &  Sobald sich zwei Spieler gefunden haben ist es möglich die Partie zu starten.\\
        \hline
        BEGRÜNDUNG: &  Es sollen keine Einzelspiele möglich sein.\\
        \hline
    \end{tabularx}

    \begin{tabularx}{\textwidth}{|l|X |} \hline
        \textbf{ID} & \textbf{NFA7} \\
        \hline
        TITEL: &  Beobachter\\
        \hline
        BESCHREIBUNG: &  Der Server erlaubt es neutralen Personen sich als Beobachter zuzuschalten und liefert ihnen den neusten Spielstand.\\
        \hline
        BEGRÜNDUNG: &  Das Spiel soll (neutralen) Beobachtern zugänglich sein.\\
        \hline
    \end{tabularx}
    
    \begin{tabularx}{\textwidth}{|l|X |} \hline
        \textbf{ID} & \textbf{NFA8} \\
        \hline
        TITEL: &  Toleranz zwischen Server und Client\\
        \hline
        BESCHREIBUNG: &  Bekommt der Server eine Nachricht von einem Client zu spät, wird die fehlende Information durch Default Informationen ersetzt.\\
        \hline
        BEGRÜNDUNG: &  Kurzzeitige VErbindungsprobleme sollen einen Client nicht gleich aus dem Spiel werfen.\\
        \hline
    \end{tabularx}

    \begin{tabularx}{\textwidth}{|l|X |} \hline
        \textbf{ID} & \textbf{NFA9} \\
        \hline
        TITEL: &  Spielerausschluss\\
        \hline
        BESCHREIBUNG: &  Nicht Standart konforme Nachrichten werden als Protokollverletzung gewehrtet und führen zum Ausschluss des Spielers.\\
        \hline
        BEGRÜNDUNG: &  So wird eine Protokolltreue erzwungen.\\
        \hline
    \end{tabularx}

    \begin{tabularx}{\textwidth}{|l|X |} \hline
        \textbf{ID} & \textbf{NFA10} \\
        \hline
        TITEL: &  Kommunikationsprotokoll- und Spielreglverstöße\\
        \hline
        BESCHREIBUNG: &  Verletzt ein SPieler das KOmmunikationsprotokoll oder die SPielregl, so wird er mittels einer Fehlermeldung aufgeklärt und vom Spielausgeschlossen.\\
        \hline
        BEGRÜNDUNG: &  Es soll verhindern, das Bugs auftretten oder durch gezielte Regelverstöße ein Absturz provoziert wird.\\
        \hline
    \end{tabularx}

    \begin{tabularx}{\textwidth}{|l|X |} \hline
        \textbf{ID} & \textbf{NFA11} \\
        \hline
        TITEL: &  Spielstandinformation\\
        \hline
        BESCHREIBUNG: &  Der Server muss Clients über den Spielstand informieren.\\
        \hline
        BEGRÜNDUNG: &  Spieler und Beobachter wissen immer bescheid.\\
        \hline
    \end{tabularx}

    \begin{tabularx}{\textwidth}{|l|X |} \hline
        \textbf{ID} & \textbf{NFA12} \\
        \hline
        TITEL: &  Client\\
        \hline
        BESCHREIBUNG: &  Instanz welche es einen einzelnen SPieler/Beobachter ermöglicht am Spiel teilzunehmen.\\
        \hline
        BEGRÜNDUNG: &  Spiel soll im Mehrspielermodus stattfinden.\\
        \hline
    \end{tabularx}

    \begin{tabularx}{\textwidth}{|l|X |} \hline
        \textbf{ID} & \textbf{NFA13} \\
        \hline
        TITEL: &  Board-Konfiguration\\
        \hline
        BESCHREIBUNG: &  Datei welche die Spielbrettinformationen enthält.\\
        \hline
        BEGRÜNDUNG: &  Jedes Spiel soll unterschieldich sein.\\
        \hline
    \end{tabularx}

    \begin{tabularx}{\textwidth}{|l|X |} \hline
        \textbf{ID} & \textbf{NFA14} \\
        \hline
        TITEL: &  KI-CLient\\
        \hline
        BESCHREIBUNG: &  Von künstlicher Intiligenz gesteireter Spieler.\\
        \hline
        BEGRÜNDUNG: &  Um vereinsamte Spieler ein Spiel zu ermöglichen.\\
        \hline
    \end{tabularx}

    \begin{tabularx}{\textwidth}{|l|X |} \hline
        \textbf{ID} & \textbf{NFA15} \\
        \hline
        TITEL: &  Editor\\
        \hline
        BESCHREIBUNG: &  Mit einem Editor werden Content und Konfigrationen des Spiels erstellt.\\
        \hline
        BEGRÜNDUNG: &  Spiel soll spielbar sein.\\
        \hline
    \end{tabularx}
    
     \begin{tabularx}{\textwidth}{|l|X |} \hline
        \textbf{ID} & \textbf{NFA16} \\
        \hline
        TITEL: &  Sprache\\
        \hline
        BESCHREIBUNG: &  Anwendersprache soll Deutsch oder Englisch sein und die Implementierungssprache Englisch.\\
        \hline
        BEGRÜNDUNG: &  Englisch ist die Sprache der Informatik.\\
        \hline
    \end{tabularx}

    \begin{tabularx}{\textwidth}{|l|X |} \hline
        \textbf{ID} & \textbf{NFA17} \\
        \hline
        TITEL: &  Implementierung\\
        \hline
        BESCHREIBUNG: &  Freiheit welche Sprache und Frameworks verwendet werden. \\
        \hline
        BEGRÜNDUNG: &  Es soll eine breitere Möglichkeit bieten und auch unterschiedliche Möglichkeiten für die unterschiedlichen Bereiche\\
        \hline
    \end{tabularx}

    \begin{tabularx}{\textwidth}{|l|X |} \hline
        \textbf{ID} & \textbf{NFA18} \\
        \hline
        TITEL: &  Plattformen\\
        \hline
        BESCHREIBUNG: &  Client und Editor müssen mindestens auf Linus, Windows oder einer Webtechnologie laufen.
        Ki und Server müssen als Docker laufen\\
        \hline
        BEGRÜNDUNG: &  So ist gewährleistet, dass alle gängigen Plattformen das Spiel spielen können.\\
        \hline
    \end{tabularx}

    \begin{tabularx}{\textwidth}{|l|X |} \hline
        \textbf{ID} & \textbf{NFA19} \\
        \hline
        TITEL: &  Netwerkkommunikation\\
        \hline
        BESCHREIBUNG: &  Die NEtzwerkkommunikation  soll über die WebSocket-Verbindung in UTF-8-Encoding erfolgen.\\
        \hline
        BEGRÜNDUNG: &  \\
        \hline
    \end{tabularx}

    \begin{tabularx}{\textwidth}{|l|X |} \hline
        \textbf{ID} & \textbf{NFA20} \\
        \hline
        TITEL: &  Dateiformat Content-und Konfigurations-Datei\\
        \hline
        BESCHREIBUNG: &  Beide Datein sollen Json Formt vorliegen.\\
        \hline
        BEGRÜNDUNG: &  Bekanntes Format.\\
        \hline
    \end{tabularx}

    \begin{tabularx}{\textwidth}{|l|X |} \hline
        \textbf{ID} & \textbf{NFA21} \\
        \hline
        TITEL: &  Namen\\
        \hline
        BESCHREIBUNG: &  Clients und KI teilen dem Server bei Anmeldung einen Namen mit.\\
        \hline
        BEGRÜNDUNG: &  So wird gewährliestet, dass für alle Spielende klar erkenbar ist, wer spielt.\\
        \hline
    \end{tabularx}

    \begin{tabularx}{\textwidth}{|l|X |} \hline
        \textbf{ID} & \textbf{NFA22} \\
        \hline
        TITEL: &  KI-Start\\
        \hline
        BESCHREIBUNG: &  Ein KI-Client muss nicht-interaktiv über die Kommandozeile in Form eines Docker-Containers gestartet werden können. Im SPiel ist die KI aber autonom.\\
        \hline
        BEGRÜNDUNG: &  Spieler können KI starten um alleine spielen zu können, sollen im SPiel aber kein zugriff auf die KI haben.\\
        \hline
    \end{tabularx}
    
    \begin{tabularx}{\textwidth}{|l|X |} \hline
        \textbf{ID} & \textbf{NFA23} \\
        \hline
        TITEL: &  KI-Inteligenzstufen\\
        \hline
        BESCHREIBUNG: &  Über eine Konfigurationsdatei oder Kommandozeilenargumente können verschiedene
Intelligenzstufen oder Strategien für die KI eingestellt werden.\\
        \hline
        BEGRÜNDUNG: &  So wird es interesanter gegen ide KI zu spielen.\\
        \hline
    \end{tabularx}

    \begin{tabularx}{\textwidth}{|l|X |} \hline
        \textbf{ID} & \textbf{NFA24} \\
        \hline
        TITEL: & Headless Komponenten\\
        \hline
        BESCHREIBUNG: & Headless Komponenten sollen mit allen Abhängigkeiten, als Docker-Image paketiert werden.\\
        \hline
        BEGRÜNDUNG: & Dadurch können die Komponenten bei teamübergreifenden Interaktionen flexibel auf unterschiedichen Rechnern gestartet werden.\\
        \hline
    \end{tabularx}

    \begin{tabularx}{\textwidth}{|l|X |} \hline
        \textbf{ID} & \textbf{NFA25} \\
        \hline
        TITEL: & Robustheit \\
        \hline
        BESCHREIBUNG: & Die Anwendung darf nicht abstürzen. Bei 100 Spielen darf maximal 1 Spiel
        aufgrund eines Fehlers abgebrochen werden. \\
        \hline
        BEGRÜNDUNG: & So soll dem Spieler ein gutes Spielerlebnis offenbart werden. \\
        \hline
    \end{tabularx}
    
\end{document}