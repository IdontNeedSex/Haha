\documentclass{uulm-assignment}

\usepackage{import}
\usepackage{tabularx}
\usepackage{listings}
\usepackage{hyperref}
\usepackage{CJKutf8}
\usepackage{graphicx}

\setboolean{showsolutions}{false}

\ifthenelse{\boolean{showsolutions}}{
    \newcommand\mitloesung{1}%
}{
    \newcommand\mitloesung{0}%
}

\hypersetup{colorlinks=false,urlcolor=uulm-in}

\faculty{Institut für Softwaretechnik und Programmiersprachen\hspace{0.05cm}}
\course{Softwaregrundprojekt}
\semester{\hspace{0.05cm} WiSe 2022/23}
\supervisor{\hspace{7.9cm} Prof. Dr. Thomas Thüm, Sabrina Böhm, Valentin Kolb, Pascal Schiessle}

\assignmenttype{}
\assignmentno{}
\title{Pflichtenheft - Battle of the Centerländ}

\newcommand{\game}{Battle of the Centerländ}

\begin{document}

    \maketitle

    \section{Allgemeine Einleitung}

    \subsection{Einleitung}
    Hier kommen die Einleitung rein.
    
    \subsection{Motivation}
    Hier kommen die Motivation rein.
    \subsection{Vision}
    Hier kommen die Vision rein.
    \subsection{Projektkontext}
    Hier kommen die Projektkontext rein.
    \section{Allgemeine Fachwissen}

   Allgemeine Fachwissen für \emph{\game} (\href{https://de.wikipedia.org/wiki/Robo_Rally}{Link zu Wikipedia}).
   
   \begin{tabularx}{\textwidth}{|l|X |} \hline
        \textbf{Begriff} & \textbf{Charakter} \\
        \hline
        BESCHREIBUNG: &  \\
        \hline
        ASPEKT: &
        \\
        \hline
        BEMERKUNG: & \\
        \hline
        BEISPIEL: & +\\
        \hline
    \end{tabularx}
    
    \section{Spezifische Anforderungen}

    Dieser Abschnitt enthält alle spezifischen Anforderungen an das System. Er bietet eine detaillierte Beschreibung des Systems und seiner Funktionen.

    \subsection{Akteure und Funktionale Anforderungen}

    Dieser Abschnitt enthält alle Anforderungen, die die grundlegenden Aktionen des Softwaresystems spezifizieren.

    \begin{tabularx}{\textwidth}{|l|X |} \hline
        \textbf{ID} & \textbf{FA1} \\
        \hline
        TITEL: & Hauptmenü \\
        \hline
        BESCHREIBUNG: & Nach dem Anwendungsstart wird dem Spieler das Hauptmenü angezeigt. Der Spieler kann folgende Aktionen im Hauptmenü ausführen:
        \begin{itemize}
            \item "Multiplayer" or "Against KI" wählen
            \item Spiel starten und auf den Spielbildschirm wechseln
            \item Sprache wählen 
            \item Die Anwendung beenden
        \end{itemize}
        \\
        \hline
        BEGRÜNDUNG: & Nachdem der Benutzer die Anwendung gestartet hat, soll nicht direkt der Spielbildschirm erscheinen.
        Der Benutzer soll die Möglichkeit haben das Spiel zu starten und das Spielmodus wählen sobald es vom Benutzer gewünscht ist.\\
        \hline
        PRIORITÄT: & ++ \\
        \hline
        ABHÄNGIGKEITEN: & FA2\\
        \hline        
    \end{tabularx}

    \begin{tabularx}{\textwidth}{|l|X |} \hline
        \textbf{ID} & \textbf{FA2} \\
        \hline
        TITEL: & Spielbildschirm \\
        \hline
        BESCHREIBUNG: & Der Spieler bekommt auf dem Spielbildschirm den aktuellen Spielzustand angezeigt.
        \\
        \hline
        BEGRÜNDUNG: & Es muss einen Bildschirm geben, auf welchem der Spielzustand angezeigt wird, da das Spiel
        sonst nicht spielbar ist. \\
        \hline
        PRIORITÄT: & ++ \\
        \hline
        ABHÄNGIGKEITEN: & FA1, FA3\\
        \hline
    \end{tabularx}

    \begin{tabularx}{\textwidth}{|l|X |} \hline
        \textbf{ID} & \textbf{FA3} \\
        \hline
        TITEL: & Spielfeld anzeigen \\
        \hline
        BESCHREIBUNG: & Das Spielfeld ist ein kartesisches Raster x * y (schachbrettartige Felder) und mit dem Editor kann man solche Spielbretter erstellen und in einer Board-Konfiguration speichern. 
        \\
        \hline
        BEGRÜNDUNG: & Es muss ein für zwei bis sechs Spielern sichtbares Spielfeld geben, damit sie spielen können.\\
        \hline
        ABHÄNGIGKEITEN: & FA2\\
        \hline
    \end{tabularx}

    \begin{tabularx}{\textwidth}{|l|X |} \hline
        \textbf{ID} & \textbf{FA4} \\
        \hline
        TITEL: & Spielfeld \\
        \hline
        BESCHREIBUNG: & Ein Spielfeld besteht aus verschiedenen Arten von Feldern, die von Charakteren betretbar sein können oder nicht. 
        \\
        \hline
        BEGRÜNDUNG: & \\
        \hline
        PRIORITÄT: & ++\\
        \hline
        ABHÄNGIGKEITEN: & FA3, FA5\\
        \hline
    \end{tabularx}

    \begin{tabularx}{\textwidth}{|l|X |} \hline
        \textbf{ID} & \textbf{FA5} \\
        \hline
        TITEL: &  Feld\\
        \hline
        BESCHREIBUNG: & Es sollte verschiedene Arten von Feldern geben, die verschiedene Eigenschaften haben. Ein Feld, auf dem ein anderer Charakter steht, ist für einen Charakter trotzdem betretbar, da Charaktere sich gegenseitig schieben können. 
        \\
        \hline
        BEGRÜNDUNG: & \\
        \hline
        PRIORITÄT: & +\\
        \hline
        ABHÄNGIGKEITEN: & FA4\\
        \hline
    \end{tabularx}
    
    \begin{tabularx}{\textwidth}{|l|X |} \hline
        \textbf{ID} & \textbf{FA} \\
        \hline
        TITEL: &  \\
        \hline
        BESCHREIBUNG: &
        \\
        \hline
        BEGRÜNDUNG: & \\
        \hline
        PRIORITÄT: & +\\
        \hline
        ABHÄNGIGKEITEN: & \\
        \hline
        AKTEUR: & \\
        \hline
    \end{tabularx}
    
    \begin{tabularx}{\textwidth}{|l|X |} \hline
        \textbf{ID} & \textbf{FA} \\
        \hline
        TITEL: &  \\
        \hline
        BESCHREIBUNG: &
        \\
        \hline
        BEGRÜNDUNG: & \\
        \hline
        PRIORITÄT: & +\\
        \hline
        ABHÄNGIGKEITEN: & \\
        \hline
        AKTEUR: & \\
        \hline
    \end{tabularx}

    \subsection{Nicht-funktionale Anforderungen}

    Dieser Abschnitt spezifiziert die nicht-funktionalen Anforderungen an das Softwaresystem.

    \begin{tabularx}{\textwidth}{|l|X |} \hline
        \textbf{ID} & \textbf{NFA1} \\
        \hline
        TITEL: & Kommunikation \\
        \hline
        BESCHREIBUNG: & Clients kommunizieren nur mit dem Server. Also findet keine Kommunikation zwischen den Clients selbst statt. \\
        \hline
        BEGRÜNDUNG: & Somit ist gewähleistet, dass der Server immer bescheid weiß. \\
        \hline
    \end{tabularx}

    \begin{tabularx}{\textwidth}{|l|X |} \hline
        \textbf{ID} & \textbf{NFA2} \\
        \hline
        TITEL: &  Server\\
        \hline
        BESCHREIBUNG: & Instanz welche eine Partie verwaltet und abwickelt.  Er ist verantwortlich die Spieleinstellungen zu Beginn des Spiels zu laden und die Spielregeln sowie das Spielbred zu verwalten.\\
        \hline
        BEGRÜNDUNG: &  Dadurch ist es möglich, das Spiel als im Mehrspielermodus zu gestalten.\\
        \hline
    \end{tabularx}

    \begin{tabularx}{\textwidth}{|l|X |} \hline
        \textbf{ID} & \textbf{NFA3} \\
        \hline
        TITEL: &  PartieKonfiguration\\
        \hline
        BESCHREIBUNG: &  Datei welche die konkreten Einstellungen einer Partie enthält\\
        \hline
        BEGRÜNDUNG: &  Bessere Übersichtlichkeit.\\
        \hline
    \end{tabularx}

    \begin{tabularx}{\textwidth}{|l|X |} \hline
        \textbf{ID} & \textbf{NFA4} \\
        \hline
        TITEL: &  Serverstart\\
        \hline
        BESCHREIBUNG: &  Ein Server muss nicht-interaktiv über die Kommandozeile in Form eines Docker-
Containers gestartet werden können \\
        \hline
        BEGRÜNDUNG: &  Anwender sollen nichts mt dem Serverstart zu tun haben und so soll ein schnellere Serververfügbarkeit gewährleistet werden.\\
        \hline
    \end{tabularx}

    \begin{tabularx}{\textwidth}{|l|X |} \hline
        \textbf{ID} & \textbf{NFA5} \\
        \hline
        TITEL: & Anzahl Clients \\
        \hline
        BESCHREIBUNG: &  Der Server ermöglicht es bis zu sechs Clients sich als MItspieler anzumelden.\\
        \hline
        BEGRÜNDUNG: &  So wird die Spielranzahl aus der Board-Konfig eingehalten.\\
        \hline
    \end{tabularx}

    \begin{tabularx}{\textwidth}{|l|X |} \hline
        \textbf{ID} & \textbf{NFA6} \\
        \hline
        TITEL: &  möglicher Spielstart\\
        \hline
        BESCHREIBUNG: &  Sobald sich zwei Spieler gefunden haben ist es möglich die Partie zu starten.\\
        \hline
        BEGRÜNDUNG: &  Es sollen keine Einzelspiele möglich sein.\\
        \hline
    \end{tabularx}

    \begin{tabularx}{\textwidth}{|l|X |} \hline
        \textbf{ID} & \textbf{NFA7} \\
        \hline
        TITEL: &  Beobachter\\
        \hline
        BESCHREIBUNG: &  Der Server erlaubt es neutralen Personen sich als Beobachter zuzuschalten und liefert ihnen den neusten Spielstand.\\
        \hline
        BEGRÜNDUNG: &  Das Spiel soll (neutralen) Beobachtern zugänglich sein.\\
        \hline
    \end{tabularx}

    \begin{tabularx}{\textwidth}{|l|X |} \hline
        \textbf{ID} & \textbf{NFA8} \\
        \hline
        TITEL: &  Toleranz zwischen Server und Client\\
        \hline
        BESCHREIBUNG: &  Bekommt der Server eine Nachricht von einem Client zu spät, wird die fehlende Information durch Default Informationen ersetzt.\\
        \hline
        BEGRÜNDUNG: &  Kurzzeitige VErbindungsprobleme sollen einen Client nicht gleich aus dem Spiel werfen.\\
        \hline
    \end{tabularx}

    \begin{tabularx}{\textwidth}{|l|X |} \hline
        \textbf{ID} & \textbf{NFA9} \\
        \hline
        TITEL: &  Spielerausschluss\\
        \hline
        BESCHREIBUNG: &  Nicht Standart konforme Nachrichten werden als Protokollverletzung gewehrtet und führen zum Ausschluss des Spielers.\\
        \hline
        BEGRÜNDUNG: &  So wird eine Protokolltreue erzwungen.\\
        \hline
    \end{tabularx}

    \begin{tabularx}{\textwidth}{|l|X |} \hline
        \textbf{ID} & \textbf{NFA10} \\
        \hline
        TITEL: &  Kommunikationsprotokoll- und Spielreglverstöße\\
        \hline
        BESCHREIBUNG: &  Verletzt ein SPieler das KOmmunikationsprotokoll oder die SPielregl, so wird er mittels einer Fehlermeldung aufgeklärt und vom Spielausgeschlossen.\\
        \hline
        BEGRÜNDUNG: &  Es soll verhindern, das Bugs auftretten oder durch gezielte Regelverstöße ein Absturz provoziert wird.\\
        \hline
    \end{tabularx}

    \begin{tabularx}{\textwidth}{|l|X |} \hline
        \textbf{ID} & \textbf{NFA11} \\
        \hline
        TITEL: &  Spielstandinformation\\
        \hline
        BESCHREIBUNG: &  Der Server muss Clients über den Spielstand informieren.\\
        \hline
        BEGRÜNDUNG: &  Spieler und Beobachter wissen immer bescheid.\\
        \hline
    \end{tabularx}

    \begin{tabularx}{\textwidth}{|l|X |} \hline
        \textbf{ID} & \textbf{NFA12} \\
        \hline
        TITEL: &  Client\\
        \hline
        BESCHREIBUNG: &  Instanz welche es einen einzelnen SPieler/Beobachter ermöglicht am Spiel teilzunehmen.\\
        \hline
        BEGRÜNDUNG: &  Spiel soll im Mehrspielermodus stattfinden.\\
        \hline
    \end{tabularx}

    \begin{tabularx}{\textwidth}{|l|X |} \hline
        \textbf{ID} & \textbf{NFA13} \\
        \hline
        TITEL: &  Board-Konfiguration\\
        \hline
        BESCHREIBUNG: &  Datei welche die Spielbrettinformationen enthält.\\
        \hline
        BEGRÜNDUNG: &  Jedes Spiel soll unterschieldich sein.\\
        \hline
    \end{tabularx}

    \begin{tabularx}{\textwidth}{|l|X |} \hline
        \textbf{ID} & \textbf{NFA14} \\
        \hline
        TITEL: &  KI-CLient\\
        \hline
        BESCHREIBUNG: &  Von künstlicher Intiligenz gesteireter Spieler.\\
        \hline
        BEGRÜNDUNG: &  Um vereinsamte Spieler ein Spiel zu ermöglichen.\\
        \hline
    \end{tabularx}

    \begin{tabularx}{\textwidth}{|l|X |} \hline
        \textbf{ID} & \textbf{NFA15} \\
        \hline
        TITEL: &  Editor\\
        \hline
        BESCHREIBUNG: &  Mit einem Editor werden Content und Konfigrationen des Spiels erstellt.\\
        \hline
        BEGRÜNDUNG: &  Spiel soll spielbar sein.\\
        \hline
    \end{tabularx}

    \begin{tabularx}{\textwidth}{|l|X |} \hline
        \textbf{ID} & \textbf{NFA16} \\
        \hline
        TITEL: &  Sprache\\
        \hline
        BESCHREIBUNG: &  Anwendersprache soll Deutsch oder Englisch sein und die Implementierungssprache Englisch.\\
        \hline
        BEGRÜNDUNG: &  Englisch ist die Sprache der Informatik.\\
        \hline
    \end{tabularx}

    \begin{tabularx}{\textwidth}{|l|X |} \hline
        \textbf{ID} & \textbf{NFA17} \\
        \hline
        TITEL: &  Implementierung\\
        \hline
        BESCHREIBUNG: &  Freiheit welche Sprache und Frameworks verwendet werden. \\
        \hline
        BEGRÜNDUNG: &  Es soll eine breitere Möglichkeit bieten und auch unterschiedliche Möglichkeiten für die unterschiedlichen Bereiche\\
        \hline
    \end{tabularx}

    \begin{tabularx}{\textwidth}{|l|X |} \hline
        \textbf{ID} & \textbf{NFA18} \\
        \hline
        TITEL: &  Plattformen\\
        \hline
        BESCHREIBUNG: &  Client und Editor müssen mindestens auf Linus, Windows oder einer Webtechnologie laufen.
        Ki und Server müssen als Docker laufen\\
        \hline
        BEGRÜNDUNG: &  So ist gewährleistet, dass alle gängigen Plattformen das Spiel spielen können.\\
        \hline
    \end{tabularx}

    \begin{tabularx}{\textwidth}{|l|X |} \hline
        \textbf{ID} & \textbf{NFA19} \\
        \hline
        TITEL: &  Netwerkkommunikation\\
        \hline
        BESCHREIBUNG: &  Die NEtzwerkkommunikation  soll über die WebSocket-Verbindung in UTF-8-Encoding erfolgen.\\
        \hline
        BEGRÜNDUNG: &  \\
        \hline
    \end{tabularx}

    \begin{tabularx}{\textwidth}{|l|X |} \hline
        \textbf{ID} & \textbf{NFA20} \\
        \hline
        TITEL: &  Dateiformat Content-und Konfigurations-Datei\\
        \hline
        BESCHREIBUNG: &  Beide Datein sollen Json Formt vorliegen.\\
        \hline
        BEGRÜNDUNG: &  Bekanntes Format.\\
        \hline
    \end{tabularx}

    \begin{tabularx}{\textwidth}{|l|X |} \hline
        \textbf{ID} & \textbf{NFA21} \\
        \hline
        TITEL: &  Namen\\
        \hline
        BESCHREIBUNG: &  Clients und KI teilen dem Server bei Anmeldung einen Namen mit.\\
        \hline
        BEGRÜNDUNG: &  So wird gewährliestet, dass für alle Spielende klar erkenbar ist, wer spielt.\\
        \hline
    \end{tabularx}

    \begin{tabularx}{\textwidth}{|l|X |} \hline
        \textbf{ID} & \textbf{NFA22} \\
        \hline
        TITEL: &  KI-Start\\
        \hline
        BESCHREIBUNG: &  Ein KI-Client muss nicht-interaktiv über die Kommandozeile in Form eines Docker-Containers gestartet werden können. Im SPiel ist die KI aber autonom.\\
        \hline
        BEGRÜNDUNG: &  Spieler können KI starten um alleine spielen zu können, sollen im SPiel aber kein zugriff auf die KI haben.\\
        \hline
    \end{tabularx}
    
    \begin{tabularx}{\textwidth}{|l|X |} \hline
        \textbf{ID} & \textbf{NFA23} \\
        \hline
        TITEL: &  KI-Inteligenzstufen\\
        \hline
        BESCHREIBUNG: &  Über eine Konfigurationsdatei oder Kommandozeilenargumente können verschiedene
Intelligenzstufen oder Strategien für die KI eingestellt werden.\\
        \hline
        BEGRÜNDUNG: &  So wird es interesanter gegen ide KI zu spielen.\\
        \hline
    \end{tabularx}

    \begin{tabularx}{\textwidth}{|l|X |} \hline
        \textbf{ID} & \textbf{NFA24} \\
        \hline
        TITEL: & Headless Komponenten\\
        \hline
        BESCHREIBUNG: & Headless Komponenten sollen mit allen Abh ̈angigkeiten, als Docker-Image paketiert werden.\\
        \hline
        BEGRÜNDUNG: & Dadurch können die Komponenten bei teamübergreifenden Interaktionen flexibel auf unterschiedichen Rechnern gestartet werden.\\
        \hline
    \end{tabularx}
    
    
    \begin{tabularx}{\textwidth}{|l|X |} \hline
        \textbf{ID} & \textbf{NFA25} \\
        \hline
        TITEL: & Robustheit \\
        \hline
        BESCHREIBUNG: & Die Anwendung darf nicht abstürzen. Bei 100 Spielen darf maximal 1 Spiel
        aufgrund eines Fehlers abgebrochen werden. \\
        \hline
        BEGRÜNDUNG: & So soll dem Spieler ein gutes Spielerlebnis offenbart werden. \\
        \hline
    \end{tabularx}
\end{document}